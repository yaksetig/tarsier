\section{Implementation}\label{sec:implementation}

\tarsier{} is implemented in Rust as a Cargo workspace of 11 crates.
Fig.~\ref{fig:arch} shows the architecture and data flow.

\begin{figure}[t]
\centering
% Tarsier Architecture Diagram
\begin{center}
\begin{tikzpicture}[
    every node/.style={rectangle, rounded corners, draw=black, very thick,
        text centered, font=\small},
    crate/.style={minimum width=2.8cm, minimum height=1.8em,
        fill=blue!8},
    artifact/.style={minimum width=2.2cm, minimum height=1.8em,
        fill=gray!15, draw=black!60, dashed},
    extool/.style={minimum width=2.2cm, minimum height=1.8em,
        fill=orange!12, draw=black!60, dotted},
    arr/.style={-{Stealth[length=2.5mm]}, thick},
    darr/.style={-{Stealth[length=2.5mm]}, thick, dashed},
]

% Row 1: Input
\node[artifact] (src) at (0, 4.5) {\texttt{.trs} file};

% Row 2: Frontend
\node[crate] (dsl)  at (0, 3.2) {tarsier-dsl};
\node[crate] (ir)   at (0, 1.9) {tarsier-ir};

% Row 3: Verification
\node[crate] (smt)    at (-2.5, 0.6) {tarsier-smt};
\node[crate] (engine) at ( 0,   0.6) {tarsier-engine};
\node[crate] (prob)   at ( 2.5, 0.6) {tarsier-prob};

% Row 4: Solvers
\node[extool] (z3)   at (-3.2, -0.7) {Z3};
\node[extool] (cvc5) at (-1.2, -0.7) {cvc5};

% Row 4: Outputs
\node[artifact] (cert)   at ( 1.0, -0.7) {Certificate};
\node[artifact] (ta)     at ( 3.2, -0.7) {\texttt{.ta} export};

% Row 5: Proof infrastructure
\node[crate] (kernel)    at (-0.5, -2.0) {tarsier-proof-kernel};
\node[crate] (certcheck) at ( 2.8, -2.0) {tarsier-certcheck};

% Row 1 side: CLI
\node[crate] (cli)   at ( 4.5, 3.2) {tarsier-cli};

% Arrows: data flow
\draw[arr] (src)    -- (dsl);
\draw[arr] (dsl)    -- node[right, draw=none, fill=none, font=\scriptsize] {AST} (ir);
\draw[arr] (ir.south) -- ++(0,-0.35) -| (engine.north);
\draw[arr] (ir.south) -- ++(0,-0.35) -| (smt.north);
\draw[arr] (ir.south) -- ++(0,-0.35) -| (prob.north);
\draw[arr] (engine) -- (smt);
\draw[arr] (prob)   -- (engine);

% Solver connections
\draw[arr] (smt.south) -- ++(0,-0.25) -| (z3.north);
\draw[arr] (smt.south) -- ++(0,-0.25) -| (cvc5.north);

% Certificate flow
\draw[darr] (engine.south) -- ++(0, -0.25) -| (cert.north);
\draw[darr] (engine.south) -- ++(0, -0.25) -| (ta.north);
\draw[arr]  (cert.south) -- ++(0, -0.25) -| (kernel.north);
\draw[arr]  (cert.south) -- ++(0, -0.25) -| (certcheck.north);

% CLI
\draw[arr] (cli) -- (dsl);
\draw[darr] (cli.south) -- ++(0, -0.9) -| (engine.north east);

% Label for TA
\node[draw=none, fill=none, font=\scriptsize\itshape, text=black!60]
    at (4.5, 1.9) {user interface};

\end{tikzpicture}
\end{center}

\caption{Architecture of \tarsier{}.  Arrows denote data flow.  The
proof kernel and certcheck are independent of the main engine, sharing
only the certificate schema.}
\label{fig:arch}
\end{figure}

\subsection{Crate Organization}

\paragraph{Frontend.}
The \texttt{tarsier-dsl} crate implements a PEG parser (via the
\texttt{pest} library) that translates \texttt{.trs} source files into an
abstract syntax tree (AST).  The \texttt{tarsier-ir} crate lowers the
AST to threshold automata, performing location expansion,
shared-variable allocation, rule generation, and property extraction
(Sect.~\ref{sec:lowering}).

\paragraph{Verification engine.}
The \texttt{tarsier-smt} crate provides a backend-agnostic SMT interface
with concrete backends for Z3~\cite{DeMoura2008} (via the \texttt{z3}
Rust crate v0.19 with static linking) and cvc5~\cite{BarbosaBBKLMMMN2022}
(via a process-based interface).  The \texttt{tarsier-engine} crate
implements BMC, $k$-induction, PDR, CEGAR~\cite{ClarkeGJLV2000}, and the
portfolio solver.  It also provides partial-order reduction, symmetry
reduction, and counterexample trace extraction.

\paragraph{Probabilistic analysis.}
The \texttt{tarsier-prob} crate implements exact hypergeometric
probability computation using arbitrary-precision rational arithmetic
(\texttt{num} crate with \texttt{BigInt}/\texttt{BigRational}).  For
committee-based protocols (e.g.\
Algorand~\cite{GiladHMVZ2017, Micali2017}), it computes the maximum
adversary bound $b_{\max}$ such that the probability of committee
corruption remains below a target $\varepsilon$, then injects this
bound into the SMT encoding.

\paragraph{Proof infrastructure.}
The \texttt{tarsier-proof-kernel} crate (Sect.~\ref{sec:certificates})
validates certificate bundles with a minimal dependency footprint.
The \texttt{tarsier-certcheck} binary replays obligations against
multiple solvers and optionally invokes proof-object checkers.

\paragraph{Auxiliary crates.}
\texttt{tarsier-conformance} validates runtime execution traces against
the protocol specification (without depending on the SMT engine).
\texttt{tarsier-lsp} provides IDE integration via the Language Server
Protocol.  \texttt{tarsier-codegen} generates verified protocol
implementations from certified specifications.

\subsection{Solver Integration}

Z3 is accessed through thread-local contexts (z3 crate v0.19 requires no
explicit context object).  Arithmetic operations use Rust's operator
overloading (\texttt{\&l + \&r}, \texttt{\&l * \&r}).  The incremental
solving interface uses push/pop scopes to extend the formula across
$k$-induction depths.

For multi-solver portfolio mode, \tarsier{} spawns parallel solver
instances and takes the first definitive result.  The portfolio strategy is
particularly effective for proof search, where Z3 and cvc5 have
complementary strengths.

\subsection{CLI and Workflow}

The \texttt{tarsier-cli} provides commands for the full verification
workflow:

\begin{itemize}
  \item \texttt{verify}: BMC safety check up to depth $k$.
  \item \texttt{prove}: Unbounded safety proof via $k$-induction or PDR.
  \item \texttt{prove-fair}: Fair-liveness proof with weak/strong fairness.
  \item \texttt{analyze}: Multi-layer analysis with graduated confidence.
  \item \texttt{certify-safety}: Generate a proof certificate bundle.
  \item \texttt{export-ta}: Export the threshold automaton in ByMC
    \texttt{.ta} format for cross-tool comparison.
\end{itemize}

Output is available in human-readable and JSON formats, enabling
integration into CI pipelines and automated regression suites.
