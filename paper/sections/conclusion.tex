\section{Conclusion}\label{sec:conclusion}

We presented \tarsier{}, a parameterized model checker for threshold-guarded
distributed protocols that advances the state of the art in three directions:
faithful network semantics via a hierarchy of four abstraction modes,
first-class cryptographic object declarations with SMT-encoded
non-forgeability, and proof-carrying verification with a minimal
independently-validating proof kernel.

Our evaluation on 44~protocol models from 14~families demonstrates that
\tarsier{} scales from minimal 4-location kernels (verified in $<$50\,ms) to
complex 1024-location models (verified in $<$2\,s), that faithful network
modes detect real bugs masked by classical counter abstraction, and that
proof certificates enable independent validation with a trusted computing
base of four library dependencies.

Several directions remain open.  First, extending the network mode hierarchy
to handle \emph{dynamic} committee membership and reconfiguration protocols
would broaden applicability to systems like Ethereum~2.0.  Second, the
current liveness analysis uses bounded lasso checking; integrating
\emph{fair termination} proofs via ranking functions or well-founded
orderings would strengthen liveness guarantees.  Third, the process-selective
mode achieves instance-exact semantics but sacrifices parameterized
abstraction; developing \emph{bounded parametric} techniques that combine
the precision of process-selective mode with the generality of symbolic
parameters is a promising research direction.  Finally, connecting
\tarsier{}'s verified specifications to certified code generation would
close the gap between formal models and deployed implementations.
