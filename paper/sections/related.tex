\section{Related Work}\label{sec:related}

\paragraph{Threshold automata and counter abstraction.}
Threshold automata were introduced by Konnov, Veith, and
Widder~\cite{KonnovVW2015} as a framework for parameterized verification of
fault-tolerant distributed algorithms.  The short counterexample
property~\cite{KonnovLVW2017} established that safety violations in threshold
automata can always be witnessed within a bounded number of steps, enabling
BMC-based verification.  The ByMC tool~\cite{ByMC2017, KonnovWidder2018}
implemented this approach with path reduction and acceleration, demonstrating
practical verification of simplified BFT protocols.  Bertrand
et~al.~\cite{BertrandKLW2021} extended the approach with decomposition
techniques for compositional verification.

\tarsier{} builds on this foundation but addresses three limitations of prior
threshold-automata tools.  First, classical counter abstraction merges all
processes in the same location, losing per-process delivery semantics;
\tarsier{}'s network mode hierarchy (Sect.~\ref{sec:semantics}) provides
progressively faithful abstractions with formally characterized soundness
transfer.  Second, ByMC provides no first-class support for quorum
certificates or threshold signatures; \tarsier{}'s cryptographic object
declarations (Sect.~\ref{sec:crypto}) encode non-forgeability and signer-set
constraints directly in SMT.  Third, ByMC requires trusting the entire
toolchain; \tarsier{}'s proof certificates (Sect.~\ref{sec:certificates}) and
minimal proof kernel reduce the trusted computing base to four library
dependencies.

\paragraph{Explicit-state model checking.}
SPIN~\cite{Holzmann2003} verifies Promela models via explicit-state
enumeration with partial-order reduction.  TLC, the model checker for
TLA+~\cite{LamportTLA2002}, explores finite instantiations of temporal-logic
specifications.  Both tools require fixing the number of processes $n$ before
verification, yielding results that are inherently bounded: verifying for
$n{=}4$ says nothing about $n{=}100$.  \tarsier{} keeps $n$, $t$, and $f$
symbolic throughout, producing proofs that hold for all instantiations
satisfying the resilience condition.  On the other hand, SPIN and TLA+ can
model richer state spaces (arbitrary data types, unbounded message buffers)
that fall outside the threshold-automata fragment.

\paragraph{Interactive and semi-automatic verification.}
Ivy~\cite{PadonMPSS2016} combines first-order logic with interactive
generalization to verify safety properties of distributed protocols.
I4~\cite{MaGJKKS2019} automates invariant inference via incremental
learning from counterexamples.  Both tools handle general-purpose distributed
protocols but require significant user guidance to discover inductive
invariants.  In contrast, \tarsier{} targets the restricted class of
threshold-guarded protocols, where the short counterexample property enables
fully automatic BMC and $k$-induction without user-supplied invariants.
This restriction is a deliberate trade-off: the class is expressive enough
to capture the core safety arguments of PBFT, HotStuff, Tendermint, Paxos,
and Algorand, while remaining decidable for automated solvers.

\paragraph{SMT-based model checking.}
Bounded model checking (BMC) via SAT was introduced by Biere
et~al.~\cite{BiereCCSZ1999}; the extension to SMT theories enables
reasoning about integers, arrays, and uninterpreted functions.
$k$-Induction~\cite{SheehySSS2000} extends BMC to unbounded proofs by
combining a base case with an inductive step.
Property-directed reachability (PDR/IC3)~\cite{Bradley2011} discovers
inductive invariants incrementally by generalizing from counterexamples to
induction.  \tarsier{} implements all three strategies (BMC,
$k$-induction, PDR) together with CEGAR~\cite{ClarkeGJLV2000} and
portfolio solving across Z3~\cite{DeMoura2008} and
cvc5~\cite{BarbosaBBKLMMMN2022}.

\paragraph{Partial-order and symmetry reduction.}
Static partial-order reduction (POR)~\cite{Peled1993, Godefroid1996} prunes
interleavings of independent transitions.  \tarsier{} adapts POR to the
counter-abstraction setting, where independence is determined by disjoint
source/target locations and absence of shared-variable conflicts.  Symmetry
reduction exploits parameter permutations in PDR cubes to prune redundant
candidates.

\paragraph{Proof-carrying verification.}
The proof-carrying code paradigm~\cite{Necula1997} separates proof
\emph{generation} from proof \emph{validation}, enabling a small, trusted
checker to verify proofs produced by a complex, untrusted compiler.
\tarsier{} applies this paradigm to model checking: the verification engine
generates proof certificates containing SMT obligations, which are
independently validated by a minimal proof kernel and optionally replayed
against multiple solvers.  The Carcara proof checker~\cite{Carcara2023}
validates solver proof objects in the Alethe format, providing an additional
layer of assurance.

\paragraph{Protocol-specific verification.}
Several works verify specific BFT protocols rather than providing a
general-purpose framework.  Notable examples include
Bracha's reliable broadcast~\cite{Bracha1987}, which we use as a running
example, and verification efforts targeting
PBFT~\cite{CastroLiskov1999}, HotStuff~\cite{YinMGRGA2019}, and
Tendermint~\cite{BuchmanKM2018}.  \tarsier{}'s DSL captures the common
structure of these protocols---roles, phases, threshold guards, and
adversary models---in a single framework amenable to automated analysis.

\paragraph{Probabilistic committee analysis.}
Algorand~\cite{GiladHMVZ2017, Micali2017} uses cryptographic sortition
to select random committees, where safety depends on the probability that a
committee contains too many Byzantine members.  \tarsier{} computes this
probability exactly using hypergeometric distributions with
arbitrary-precision rational arithmetic, determining the maximum adversary
bound $b_{\max}$ such that the corruption probability remains below a target
$\varepsilon$.  This bound is then injected into the SMT encoding, yielding
a \emph{probabilistically safe} verdict with a quantified failure
probability.
